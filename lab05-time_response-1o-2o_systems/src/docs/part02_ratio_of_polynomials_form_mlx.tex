% This LaTeX was auto-generated from MATLAB code.
% To make changes, update the MATLAB code and export to LaTeX again.

\documentclass{article}

\usepackage[utf8]{inputenc}
\usepackage[T1]{fontenc}
\usepackage{lmodern}
\usepackage{graphicx}
\usepackage{color}
\usepackage{hyperref}
\usepackage{amsmath}
\usepackage{amsfonts}
\usepackage{epstopdf}
\usepackage[table]{xcolor}
\usepackage{matlab}

\sloppy
\epstopdfsetup{outdir=./}
\graphicspath{ {./part02_ratio_of_polynomials_form_mlx_images/} }

\begin{document}

\matlabtitle{Part 2 $-$ Rational of polynomials form}


\matlabheading{1. Conversion using Matlab}

\begin{par}
\begin{flushleft}
The transfer function from the state-space representation
\end{flushleft}
\end{par}

\begin{matlaboutput}
Mtf =
 
                         -s
  -------------------------------------------------
  s^6 - s^5 - s^4 - 2 s^3 + 2 s^2 + 2 s + 2.053e-16
 
Continuous-time transfer function.
\end{matlaboutput}

\matlabheading{2. Equation for transfer functions}

\begin{matlaboutput}
T =
 
                   -s - 1.307e-16
  -------------------------------------------------
  s^6 - s^5 - s^4 - 2 s^3 + 2 s^2 + 2 s - 2.435e-16
 
Continuous-time transfer function.
\end{matlaboutput}

\matlabheading{Coefficient of determination $R^2$}

\begin{par}
\begin{flushleft}
To compare the transfer functions, let's find the $R^2$ value of all coefficients.
\end{flushleft}
\end{par}


\begin{par}
\begin{flushleft}
Then the R\textasciicircum{}2
\end{flushleft}
\end{par}

\begin{matlaboutput}
R2 = 1.0000
\end{matlaboutput}
\begin{matlaboutput}
shows that the coefficients have a strong correlation
\end{matlaboutput}
\begin{matlaboutput}
and are therefore equivalent.
\end{matlaboutput}
\end{document}
