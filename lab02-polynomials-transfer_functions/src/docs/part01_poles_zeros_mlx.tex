% This LaTeX was auto-generated from MATLAB code.
% To make changes, update the MATLAB code and export to LaTeX again.

\documentclass{article}

\usepackage[utf8]{inputenc}
\usepackage[T1]{fontenc}
\usepackage{lmodern}
\usepackage{graphicx}
\usepackage{color}
\usepackage{hyperref}
\usepackage{amsmath}
\usepackage{amsfonts}
\usepackage{epstopdf}
\usepackage[table]{xcolor}
\usepackage{matlab}

\sloppy
\epstopdfsetup{outdir=./}
\graphicspath{ {./part01_poles_zeros_mlx_images/} }

\begin{document}

\matlabtitle{Part 1 $-$ Poles and zeros}


\matlabheading{1ab. Roots}

\begin{par}
\begin{flushleft}
Calculate the roots of each of the following polynomials
\end{flushleft}
\end{par}

\begin{matlabsymbolicoutput}
\hskip1em $\displaystyle P_1 =s^6 +s^5 +2\,s^4 +8\,s^3 +7\,s^2 +15\,s+12$
\hskip1em $\displaystyle P_2 =s^6 +s^5 +4\,s^4 +3\,s^3 +7\,s^2 +15\,s+18$
\end{matlabsymbolicoutput}


\begin{par}
\begin{flushleft}
The roots for each polynomial are
\end{flushleft}
\end{par}

\begin{matlabtableoutput}
{
\begin{tabular} {|c|c|c|c|}\hline
\mlcell{ } & \mlcell{CartesianForm} & \mlcell{r} & \mlcell{thetaDeg} \\ \hline
\mlcell{1} & \mlcell{0.9979 + 1.6070i} & \mlcell{1.8916} & \mlcell{58.1624} \\ \hline
\mlcell{2} & \mlcell{0.9979 - 1.6070i} & \mlcell{1.8916} & \mlcell{301.8376} \\ \hline
\mlcell{3} & \mlcell{-1.8615 + 0.0000i} & \mlcell{1.8615} & \mlcell{180} \\ \hline
\mlcell{4} & \mlcell{-0.1302 + 1.4299i} & \mlcell{1.4358} & \mlcell{95.2009} \\ \hline
\mlcell{5} & \mlcell{-0.1302 - 1.4299i} & \mlcell{1.4358} & \mlcell{264.7991} \\ \hline
\mlcell{6} & \mlcell{-0.8739 + 0.0000i} & \mlcell{0.8739} & \mlcell{180} \\ 
\hline
\end{tabular}
}
\end{matlabtableoutput}
\begin{matlabtableoutput}
{
\begin{tabular} {|c|c|c|c|}\hline
\mlcell{ } & \mlcell{CartesianForm} & \mlcell{r} & \mlcell{thetaDeg} \\ \hline
\mlcell{1} & \mlcell{1.0375 + 1.3227i} & \mlcell{1.6811} & \mlcell{51.8922} \\ \hline
\mlcell{2} & \mlcell{1.0375 - 1.3227i} & \mlcell{1.6811} & \mlcell{308.1078} \\ \hline
\mlcell{3} & \mlcell{-0.4632 + 1.9333i} & \mlcell{1.9880} & \mlcell{103.4723} \\ \hline
\mlcell{4} & \mlcell{-0.4632 - 1.9333i} & \mlcell{1.9880} & \mlcell{256.5277} \\ \hline
\mlcell{5} & \mlcell{-1.0743 + 0.6764i} & \mlcell{1.2695} & \mlcell{147.8047} \\ \hline
\mlcell{6} & \mlcell{-1.0743 - 0.6764i} & \mlcell{1.2695} & \mlcell{212.1953} \\ 
\hline
\end{tabular}
}
\end{matlabtableoutput}

\matlabheading{2. Polynomial form}

\begin{par}
\begin{flushleft}
Calculate the polynomial form and roots of
\end{flushleft}
\end{par}

\begin{matlabsymbolicoutput}
\hskip1em $\displaystyle P_3 ={\left(s-1\right)}\,{\left(s-2\right)}\,{\left(s+2\right)}\,{\left(s+3\right)}\,{\left(s+4\right)}\,{\left(s+5\right)}$
\end{matlabsymbolicoutput}

\begin{par}
\begin{flushleft}
The polynomial form is
\end{flushleft}
\end{par}

\begin{matlabsymbolicoutput}
P3\_s = 

\hskip1em $\displaystyle s^6 +11\,s^5 +31\,s^4 -31\,s^3 -200\,s^2 -52\,s+240$
\end{matlabsymbolicoutput}

\begin{par}
\begin{flushleft}
The roots of the polynomial are
\end{flushleft}
\end{par}

\begin{matlaboutput}
P3_roots = 6x1    
   -5.0000
   -4.0000
   -3.0000
   -2.0000
    2.0000
    1.0000

\end{matlaboutput}

\matlabheading{3a. Converting to polynomial numerator and denominator.}

\begin{par}
\begin{flushleft}
Represent
\end{flushleft}
\end{par}

\begin{matlaboutput}
G1 =
 
  9 (s+2) (s+3) (s+8) (s-6)
  --------------------------
  s (s+7) (s+10) (s-3) (s-2)
 
Continuous-time zero/pole/gain model.
\end{matlaboutput}

\begin{par}
\begin{flushleft}
using polynomials in the numerator and denominator.
\end{flushleft}
\end{par}

\begin{par}
\begin{flushleft}
In polynomial numerator and denominator, the transfer function
\end{flushleft}
\end{par}

\begin{matlaboutput}
G1_tf =
 
  9 s^4 + 63 s^3 - 288 s^2 - 2052 s - 2592
  ----------------------------------------
   s^5 + 12 s^4 - 9 s^3 - 248 s^2 + 420 s
 
Continuous-time transfer function.
\end{matlaboutput}

\matlabheading{3b. Converting to zero-pole-gain form.}

\begin{par}
\begin{flushleft}
Represent 
\end{flushleft}
\end{par}

\begin{matlaboutput}
G2 =
 
         s^4 + 17 s^3 + 99 s^2 + 223 s + 140
  -------------------------------------------------
  s^5 + 32 s^4 + 363 s^3 + 2092 s^2 + 5052 s + 4320
 
Continuous-time transfer function.
\end{matlaboutput}

\begin{par}
\begin{flushleft}
using factored forms of the polynomials in the numerator and denominator.
\end{flushleft}
\end{par}

\begin{par}
\begin{flushleft}
In zero-pole-gain form, the transfer function
\end{flushleft}
\end{par}

\begin{matlaboutput}
G2_zpk =
 
                 (s+7) (s+5) (s+4) (s+1)
  ------------------------------------------------------
  (s+16.79) (s^2 + 4.097s + 4.468) (s^2 + 11.12s + 57.6)
 
Continuous-time zero/pole/gain model.
\end{matlaboutput}

\matlabheading{4abc. Partial fraction expansion}

\begin{par}
\begin{flushleft}
Calculate the partial fraction expansion of each of the following transfer functions.
\end{flushleft}
\end{par}

\begin{par}
$$G_3 =\frac{5(s+2)}{s(s^2 +8s+15)},$$ $$G_4 =\frac{5(s+2)}{s(s^2 +6s+9)},$$ $$G_5 =\frac{5(s+2)}{s(s^2 +6s+34)},$$
\end{par}

\begin{par}
\begin{flushleft}
which have the zero-pole-gain forms
\end{flushleft}
\end{par}

\begin{matlaboutput}
G3 =
 
     5 (s+2)
  -------------
  s (s+5) (s+3)
 
Continuous-time zero/pole/gain model.
\end{matlaboutput}
\begin{matlaboutput}
G4 =
 
   5 (s+2)
  ---------
  s (s+3)^2
 
Continuous-time zero/pole/gain model.
\end{matlaboutput}
\begin{matlaboutput}
G5 =
 
       5 (s+2)
  -----------------
  s (s^2 + 6s + 34)
 
Continuous-time zero/pole/gain model.
\end{matlaboutput}

\begin{par}
\begin{flushleft}
The partial fraction expansions are
\end{flushleft}
\end{par}


\begin{par}
\begin{flushleft}
Well, we see that each of their residues (column \#1), poles (column \#2 in RP matrix), and their direct functions (K)
\end{flushleft}
\end{par}

\begin{matlaboutput}
G3_RP = 3x2    
   -1.5000   -5.0000
    0.8333   -3.0000
    0.6667         0

\end{matlaboutput}
\begin{matlaboutput}
G3_K =

  0x1 empty double column vector
\end{matlaboutput}
\begin{matlaboutput}
G4_RP = 3x2    
   -1.1111   -3.0000
    1.6667   -3.0000
    1.1111         0

\end{matlaboutput}
\begin{matlaboutput}
G4_K =

  0x1 empty double column vector
\end{matlaboutput}
\begin{matlaboutput}
G5_RP = 3x2 complex    
  -0.1471 - 0.4118i  -3.0000 + 5.0000i
  -0.1471 + 0.4118i  -3.0000 - 5.0000i
   0.2941 + 0.0000i   0.0000 + 0.0000i

\end{matlaboutput}
\begin{matlaboutput}
G5_K =

  0x1 empty double column vector
\end{matlaboutput}

\begin{par}
\begin{flushleft}
Thus the partial fraction expansions
\end{flushleft}
\end{par}

\begin{matlabsymbolicoutput}
G3\_partial = 

\hskip1em $\displaystyle \frac{5}{6\,{\left(s+3\right)}}-\frac{3}{2\,{\left(s+5\right)}}+\frac{2}{3\,s}$
\end{matlabsymbolicoutput}
\begin{matlabsymbolicoutput}
G4\_partial = 

\hskip1em $\displaystyle \frac{5}{3\,{{\left(s+3\right)}}^2 }-\frac{10}{9\,{\left(s+3\right)}}+\frac{10}{9\,s}$
\end{matlabsymbolicoutput}
\begin{matlabsymbolicoutput}
G5\_partial = 

\hskip1em $\displaystyle \frac{5}{17\,s}+\frac{-\frac{5}{34}-\frac{7}{17}\,\mathrm{i}}{s+3-5\,\mathrm{i}}+\frac{-\frac{5}{34}+\frac{7}{17}\,\mathrm{i}}{s+3+5\,\mathrm{i}}$
\end{matlabsymbolicoutput}

\matlabheading{complexTable(complex)}

\begin{par}
\begin{flushleft}
Creates a table showing the Cartesian forms, magnitudes and angles (in [0, 360) [deg]) of the given complex numbers.
\end{flushleft}
\end{par}

\matlabheading{Input Arguments}

\begin{par}
\begin{flushleft}
\textbf{complex} : double = vector of complex numbers
\end{flushleft}
\end{par}

\matlabheading{Output Arguments}

\begin{par}
\begin{flushleft}
\textbf{complexTable} : table (3-columns) = the table of the Cartesian forms, magnitudes and angles (in [0, 360) [deg]) of each complex numbers
\end{flushleft}
\end{par}

\matlabheading{angle360(vector)}

\begin{par}
\begin{flushleft}
Gets an angle from a vector of complex numbers in the domain of [0, 360) [deg].
\end{flushleft}
\end{par}

\matlabheading{Input Arguments}

\begin{par}
\begin{flushleft}
\textbf{vector} : double = representing the vector of complex numbers
\end{flushleft}
\end{par}

\matlabheading{Output Arguments}

\begin{par}
\begin{flushleft}
\textbf{acc} : the vector of arrays in the domain of [0, 360) [deg
\end{flushleft}
\end{par}

\matlabheading{conv\_rows(matrix)}

\begin{par}
\begin{flushleft}
Convolves the rows of a matrix into a row vector.
\end{flushleft}
\end{par}

\matlabheading{Input Arguments}

\begin{par}
\begin{flushleft}
\textbf{T} : double = 2D array whose rose to convolve
\end{flushleft}
\end{par}

\matlabheading{Output Arguments}

\begin{par}
\begin{flushleft}
\textbf{acc} : the resulting convolved row vector
\end{flushleft}
\end{par}

\end{document}
