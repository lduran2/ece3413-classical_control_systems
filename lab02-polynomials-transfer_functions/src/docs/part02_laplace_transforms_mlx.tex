% This LaTeX was auto-generated from MATLAB code.
% To make changes, update the MATLAB code and export to LaTeX again.

\documentclass{article}

\usepackage[utf8]{inputenc}
\usepackage[T1]{fontenc}
\usepackage{lmodern}
\usepackage{graphicx}
\usepackage{color}
\usepackage{hyperref}
\usepackage{amsmath}
\usepackage{amsfonts}
\usepackage{epstopdf}
\usepackage[table]{xcolor}
\usepackage{matlab}

\sloppy
\epstopdfsetup{outdir=./}
\graphicspath{ {./part02_laplace_transforms_mlx_images/} }

\begin{document}

\matlabtitle{Part 2 $-$ Laplace transforms}


\matlabheading{1. The Laplace transform}

\begin{par}
\begin{flushleft}
Find the Laplace transform of
\end{flushleft}
\end{par}

\begin{par}
$$f(t)=0.0075-0.00034e^{-2.5t} cos(22t)+0.087e^{-2.5t} sin(22t)-0.0072e^{-8t} .$$
\end{par}

\begin{matlabsymbolicoutput}
f\_t = 

\hskip1em $\displaystyle \frac{87\,\sin \left(22\,t\right)\,{\mathrm{e}}^{-\frac{5\,t}{2}} }{1000}-\frac{17\,\cos \left(22\,t\right)\,{\mathrm{e}}^{-\frac{5\,t}{2}} }{50000}-\frac{9\,{\mathrm{e}}^{-8\,t} }{1250}+\frac{3}{400}$
\end{matlabsymbolicoutput}

\begin{par}
\begin{flushleft}
The Laplace transform of $f(t)$,
\end{flushleft}
\end{par}

\begin{matlabsymbolicoutput}
F = 

\hskip1em $\displaystyle \frac{3}{400\,s}-\frac{9}{1250\,{\left(s+8\right)}}-\frac{17\,{\left(s+\frac{5}{2}\right)}}{50000\,{\left({{\left(s+\frac{5}{2}\right)}}^2 +484\right)}}+\frac{957}{500\,{\left({{\left(s+\frac{5}{2}\right)}}^2 +484\right)}}$
\end{matlabsymbolicoutput}


\matlabheading{2. The inverse of the Laplace transform}

\begin{par}
\begin{flushleft}
Find the inverse Laplace transform of
\end{flushleft}
\end{par}

\begin{matlaboutput}
F =
 
     2 (s+3) (s+5) (s+7)
  -------------------------
  s (s-8) (s^2 + 10s + 100)
 
Continuous-time zero/pole/gain model.
\end{matlaboutput}

\begin{par}
\begin{flushleft}
We can represent the transfer function in Matlab from its roots
\end{flushleft}
\end{par}

\begin{matlabsymbolicoutput}
F\_s = 

\hskip1em $\displaystyle \frac{2\,{\left(s+3\right)}\,{\left(s+5\right)}\,{\left(s+7\right)}}{s\,{\left({{\left(s+5\right)}}^2 +75\right)}\,{\left(s-8\right)}}$
\end{matlabsymbolicoutput}

\begin{par}
\begin{flushleft}
Then, the inverse Laplace transform of $F(s)$,
\end{flushleft}
\end{par}

\begin{matlabsymbolicoutput}
f\_t = 

\hskip1em $\displaystyle \frac{2145\,{\mathrm{e}}^{8\,t} }{976}+\frac{79\,{\mathrm{e}}^{-5\,t} \,{\left(\cos \left(5\,\sqrt{3}\,t\right)+9\,\sqrt{3}\,\sin \left(5\,\sqrt{3}\,t\right)\right)}}{1220}-\frac{21}{80}$
\end{matlabsymbolicoutput}

\matlabheading{factorFromComplexRoot(complexRoot, s)}

\begin{par}
\begin{flushleft}
Returns the polynomial factor $P(s)$ that when fixed to 0, gives the given \textbf{complexRoot} for polynomial variable \textbf{s}.
\end{flushleft}
\end{par}

\matlabheading{Input Arguments}

\begin{par}
\begin{flushleft}
\textbf{complexRoot} : double = root for which to find the polynomial
\end{flushleft}
\end{par}

\matlabheading{Output Arguments}

\begin{par}
\begin{flushleft}
\textbf{factor} : the factor polynomial $P(s)$ that equals 0 at \textbf{s == complexRoot}
\end{flushleft}
\end{par}

\end{document}
