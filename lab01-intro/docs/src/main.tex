\documentclass[12pt]{article}
\usepackage[utf8]{inputenc}
\usepackage{amsmath}

\title{ECE 3413 Lab 01\\*Introduction to Matlab and Simulink}
\author{Leomar Dur\'an}
\date{$1^{\text{st}}$ February 2023}

\usepackage{minted}

\usepackage{mathtools}%
\DeclarePairedDelimiter\brao()%
\DeclarePairedDelimiter\brac[]%
\DeclarePairedDelimiter\braco[)%
\DeclarePairedDelimiter\Brac\{\}%
\DeclarePairedDelimiter\norm\lVert\rVert%
\DeclarePairedDelimiter\piecefn\{.
\DeclarePairedDelimiter\evalat.|

\begin{document}

\maketitle

\section{Introduction}

The purpose of this experiment is to serve as a practical introduction to Matlab and Simulink and their graphical user interface.

Matlab is a linear algebra tool, and Simulink is a tool for modeling systems using block diagrams.
They may each be used for modeling control systems among other types of systems.

In this example, we use Matlab to model a polynomial as a row vector of coefficients.
We also use Matlab in conjunction with Simulink to prepare, model and contrast $2$ control systems.

\section{Procedure}

\subsection{Roots and corresponding phase angles of a polynomial}

In this part of the lab, we model a polynomial in Matlab.
This is done by using a row vector.
Of the operations that can be performed on a row vector, $2$ are the \mintinline{matlab}{conv} and \mintinline{matlab}{roots} functions.

\subsubsection{Convolution}

The \mintinline{matlab}{conv} function is used to convolve two vectors representing polynomial multiplication.

For example, we have the polynomials
\[
    \begin{gathered}
        P_1\brao*s = s^2 + 10s + 24, \\*
        P_2\brao*s = s^4 + 26s^3 + 231s^2 + 766s + 560. \\*
    \end{gathered}
\]

These may be represented by the row vectors
\[
    \begin{gathered}
        \vec{v}_1 := \brac*{\begin{matrix} 1 & 10 & 24 \\* \end{matrix}}, \\*
        \vec{v}_2 := \brac*{\begin{matrix} 1 & 26 & 231 & 766 & 560 \\* \end{matrix}}. \\*
    \end{gathered}
\]

Then the convolution $\mathbf{v}_1 \ast \mathbf{v}_2$ represents the product $\brao{P_1 P_2}\brao*s$ as follows

\begin{minted}{matlab}
       [   1  26 231 766 560 ]
[ 24  10   1]                         = 1
    [ 24  10   1]                     = 10+26 = 36
        [ 24  10   1]                 = 24 + 260 + 231 = 515
            [ 24  10   1]             = 624 + 2310 + 766 = 3700
                [ 24  10   1]         = 5544 + 7660 + 560 = 13764
                    [ 24  10   1]     = 18384 + 5600 = 23984
                        [ 24  10   1] = 13440
-----------------------------------------------------------------
[1 10 24]*[1 26 231 766 560] = [1 36 515 3700 13764 23984 13440],
\end{minted}

representing the polynomial
\[
    P\brao*s = s^6 + 36s^5 + 515s^4 + 3700s^3 + 13764s^2 + 23984s + 13440.
\]

\subsubsection{The roots of the polynomial}

The \mintinline{matlab}{roots} function in Matlab returns the roots of the polynomial $P\brao*s$, that is, the values of $s$ s.t. $P\brao*s = 0$.

\subsection{Transfer functions}

In part 2 of the lab, we model a transfer function and modify one of its coefficients to produce a new output.

We use Matlab to set up the parameters for the transfer functions before using Simulink to model them because Matlab allows for a cleaner interface to set up the parameters.

Additionally, we can use Matlab to perform sanity checks along the way because it echos the value of every statement that is not suppressed by the semicolon character (\mintinline{matlab}{;}).

For example, we can echo the poles, zeros, numerators and denominators of the transfer functions to ensure that we set them up correctly.

\subsubsection{Transfer functions in Matlab}

Matlab has transfer function objects.
These can be represented by objects as a ratio of polynomials using the \mintinline{matlab}{tf} function,
or as their zeros, poles and gain using \mintinline{matlab}{zpk}.
In this lab, we make use of the \mintinline{matlab}{tf} objects.

\subsubsection{Poles}

The poles are the roots of the denominator of the transfer function.
Matlab has the function \mintinline{matlab}{pole},
which accepts a transfer function and returns it poles.

\subsubsection{Zeros}

The zeros are the roots of the numerator of the transfer function.
Matlab has the function \mintinline{matlab}{zero},
which accepts a transfer function and returns it zero.

\subsubsection{The modified transfer function}

We are given the transfer function
\[
    H\brao*s = \frac{s^2 + 10s + 24}{s^4 + 26s^3 + 231s^2 + 766s + 560}.
\]

I have modified it by negating its pole with the minimum magnitude.

\end{document}
